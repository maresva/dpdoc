\chapter{Testování}
	S cílem zajistit co největší spolehlivost a tedy i kvalitu tohoto nástroje, byla knihovna i uživatelská aplikace řádně otestována. Nástroj byl otestován pomocí automatizovaných jednotkových testů, ale také manuálním testováním.
	
	\section{Jednotkové testy}
		V rámci knihovny byla vytvořena řada jednotkových testů. Tyto testy ověřují různé aspekty nástroje a byly použity pro zajištění spolehlivosti výsledného produktu, ale také jako kontrola během vývoje. Testy jsou zaměřeny zejména na extrakci kontraktů, kde se snaží analyzovat kontrakty z uměle vytvořených testovacích dat (viz níže). Jsou zde také testy pro ověření správnosti porovnávacího nástroje a modulu pro tokenizaci zdrojových souborů. Celkem bylo implementováno 55 jednotkových testů různé komplexity.
		
		\subsection{Ukázka jednotkového testu}
			Následuje ukázka jednotkového testu \texttt{testGuava()}, který se nachází ve třídě \texttt{ContractExtractorGuavaTest}. Tento test je použit v projektu a zde je zobrazen formou zjednodušeného Java kódu. Testovací soubor je pak zobrazen níže a jeho obsah je téměř shodný s reálným souborem. Jedná se o soubor \texttt{testFiles/Guava/TestGuava.java}. Ostatní testy byly realizovány obdobným způsobem.
			
			\subsubsection{Testovací soubor \texttt{TestGuava.java}}
				Zde je zobrazen obsah testovacího souboru. Jak je vidět jedná se o jednoduchý kód, který ověřuje, zda vstupní parametr \texttt{x} není \texttt{null} pomocí knihovny Guava Preconditions.\\\\	
				\- \- \- \- \- \- \texttt{public class TestGuava \{}\\
				\- \- \- \- \- \- \- \- \- \texttt{public void guavaTest(String x)\{}\\
    			\- \- \- \- \- \- \- \- \- \- \- \- \texttt{checkNotNull(x);}\\
				\- \- \- \- \- \- \- \- \- \texttt{\}}\\
				\- \- \- \- \- \- \texttt{\}}        	
			
			\subsubsection{Zjednodušený kód testu}
				Zde je zjednodušený kód jednotkového testu, který ověřuje správnost extrakce kontraktu typu Guava ze souboru uvedeného výše.\\\\
				\- \- \- \- \- \- \texttt{\textcolor{pgrey}{// Získání reprezentace kontraktu formou JavaFile}}\\ 
				\- \- \- \- \- \- \texttt{JavaFile javaFile = getTestJavaFile("TestGuava.java");}\\
				\- \- \- \- \- \- \texttt{\textcolor{pgrey}{// Test očekává právě jeden kontrakt}}\\ 
				\- \- \- \- \- \- \texttt{assertEquals(1, javaFile.getContractsSize());}\\\\
				\- \- \- \- \- \- \texttt{\textcolor{pgrey}{// Uložení jediného kontraktu do proměnné}}\\ 
				\- \- \- \- \- \- \texttt{Contract c = javaFile.getFirstContract();}\\\\
				\- \- \- \- \- \- \texttt{\textcolor{pgrey}{// Ověření, zda daný kontrakt je typu Guava}}\\
				\- \- \- \- \- \- \texttt{assertEquals(ContractType.GUAVA, c.getContractType());}\\
				\- \- \- \- \- \- \texttt{\textcolor{pgrey}{// Ověření zda se jedná o vstupní podmínku}}\\
				\- \- \- \- \- \- \texttt{assertEquals(ConditionType.PRE, c.getConditionType());}\\
				\- \- \- \- \- \- \texttt{\textcolor{pgrey}{// Ověření, zda se shoduje celý výraz}}\\
				\- \- \- \- \- \- \texttt{assertEquals("checkNotNull(x);", c.getCompleteExpr());}\\
				\- \- \- \- \- \- \texttt{\textcolor{pgrey}{// Ověření, zda se shoduje použitá funkce}}\\
				\- \- \- \- \- \- \texttt{assertEquals("checkNotNull", c.getFunction());}\\
				\- \- \- \- \- \- \texttt{\textcolor{pgrey}{// Ověření, zda se shoduje logický výraz}}\\
				\- \- \- \- \- \- \texttt{assertEquals("x", c.getExpression());}\\
				\- \- \- \- \- \- \texttt{\textcolor{pgrey}{// Ověření, zda nebyl nalezen žádný argument}}\\
				\- \- \- \- \- \- \texttt{assertEquals(0, contract.getArguments().size());}\\\\
				
				Jak je vidět, i přesto, že se jedná o jednoduchý kontrakt ve stručném souboru, test obsahuje mnoho asercí. Z tohoto důvodu jsou testy realizované v projektu parametrizované pomocí parametrů a jsou využity pomocné metody. Zde byl test pro lepší čitelnost realizován tímto způsobem bez pomocných metod.
				
		
	\section{Funkční testování}
		Vzhledem k tomu, že uživatelská aplikace je plně závislá na vytvořené knihovně, je možné knihovnu testovat zároveň s uživatelskou aplikací. Ta byla testována pomocí manuálního testování tím, že byly ověřovány různé testovací případy.
			
	\section{Testovací data}
		Vzhledem k povaze nástroje bylo třeba disponovat testovacími daty pro kontrolu funkčnosti. Pro tento účelem jsem využil skutečná data dodaná vedoucím práce, ale také jsem si některá sám vytvořil.
		
		\subsection{Skutečná data}
			Dodaná data obsahovala zdrojové soubory několika projektů různých verzí. Během testování jsem primárně využil samotného projektu Guava, který obsahuje velké množství kontraktů realizovaných pomocí JSR305 a Guava Preconditions. Knihovna se vyskytuje v mnoha verzích a obsahuje řadu souborů. Z tohoto důvodu byla vhodná nejen pro testování extrakce kontraktů, ale také pro ověření správnosti porovnávání. V mnohem menší míře byly také použity projekty Annotations, JSR305 a Reflections. Důvodem byl nízký počet kontraktů zkoumaného typu.\\
			
			Skutečná data byla použita při manuálním testování nástroje. Díky své obsáhlosti mi nepřišla data vhodná pro automatizované testy. Mnoho jednotkových testů však bylo inspirováno použitím reálných reprezentací kontraktů nalezených v těchto datech.
			
		\subsection{Syntaktická data}
			V reálných datech je obvyklé, že se využije jen část funkčnosti, kterou daná knihovna nabízí. Aby byl nástroj řádně otestován, uměle jsem vytvořil data, která pokryla různé možnosti použití kontraktů. To jak vytvořená data je dobře vidět v ukázce jednotkového testu výše. Zde se jedná o jednoduchý test vhodný pro demonstraci v dokumentu, jiná testovací data jsou více komplexní. Všechna tato data se nacházejí v projektu ve zdrojích ve složce \texttt{testFiles}. Zde jsou rozděleny do složek podle oblasti testování (\texttt{comparator} pro porovnávání kontraktů, \texttt{extractor} pro extrakci kontraktů a \texttt{parser} pro analýzu jazyka Java). Tato data jsou umístěna na CD ve složce \texttt{src/ContractParser/src/main/resources/testFiles}.		 
		 
	\section{Výsledky testů}
		Ve finální verzi nástroje byly všechny testy úspěšně vykonány. Testy byly aktivně využívány v závěrečné fázi vývoje, kdy byly používány k zjištění, zda je stále zachována funkčnost nástroje po provedených změnách. 
