\chapter{Testování}
	S cílem zajistit co největší spolehlivost a tedy i kvalitu tohoto nástroje, byla knihovna i uživatelská aplikace řádně otestována. Nástroj byl otestován pomocí automatizovaných jednotkových testů, ale také manuálním testováním.
	
	\section{Jednotkové testy}
		V rámci knihovny byla vytvořena řada jednotkových testů. Tyto testy ověřují různé testovací případy a byly použity pro zajištění spolehlivosti, ale také jako kontrola během vývoje. Testy jsou zaměřeny zejména na extrakci kontraktů, kde se snaží analyzovat kontrakty z uměle vytvořených testovacích dat (viz níže). Jsou zde také testy pro ověření správnosti porovnávacího nástroje a také modulu pro tokenizaci zdrojových souborů. Celkem bylo implementováno 55 jednotkových testů různé komplexity.
		
	\section{Funkční testování}
		Vzhledem k tomu, že uživatelská aplikace je plně závislá na vytvořené knihovně, je možné tuto knihovnu testovat zároveň s uživatelskou aplikací. Ta byla testována pomocí manuálního testování tím, že byly ověřovány různé testovací případy. Následuje seznam těchto případů, pro které byla aplikace úspěšně otestována.
		
		\subsection{Testovací případy}	
			Zde zjednodušený popis testovacích případů, pro které byla uživatelská aplikace testována. Jednotlivá tlačítka a operace zmíněné v rámci popisu testovacích případů jsou podrobně popsány v příloze A Uživatelská příručka.
			
			\subsubsection{Spuštění aplikace}
				Po spuštění	aplikace bez parametrů v pořádku naběhne grafická část aplikace. Ikony tlačítek i celého okna aplikace by měly být zobrazeny.
				
			\subsubsection{Přidání souborů - přidání prvních souborů}
				Po stisknutí tlačítka \emph{Přidat soubory} naběhne okno pro výběr souborů. Zde je možné vybrat více souborů a je možné vybrat pouze soubory \texttt{.java} nebo \texttt{.class}.	 Po vybrání souborů naběhne načítací okno zobrazující průběh extrakce kontraktů (v případě malého množství souborů a/nebo vysokého výkonu se okno zobrazí jen velmi krátce).\\
				
				Po přidání souborů se vybrané soubory objeví v levém seznamu souborů (pokud není vybrán filtr \emph{Show Non Contract Objects}, soubory bez kontraktů se nezobrazí). Na základě toho se aktualizuje popisek pod seznamem (\emph{Files selected}, který by měl nyní jako druhé číslo ukazovat nový počet souborů. Dále se aktualizují údají v právé části \emph{Global Statistics}. Počet souborů i ostatní informace odpovídá přidaným datům přidaných souborů. Také se zpřístupní tlačítko \emph{Select All}, respektive \emph{Deselect All}. Do konzole je také vypsáno, kolik souborů se přidalo (celkové číslo zahrnuje i soubory skryté filtrem).\\
				
				Jestliže je okno je zavřeno tlačítkem \emph{Storno} či křížkem, okno se schová a aplikace pokračuje v běhu bez jakékoliv chyby. 
				
			\subsubsection{Přidání souborů - přidání dalších souborů}
				Každé další přidání souborů funguje stejným způsobem jako to první. Pokud se ale pokusíme přidat soubory, které již v seznamu jsou, opětovně přidány nejsou a také nebudou zahrnuty do počtu přidaných souborů ve výpisu do konzole.
				
			\subsubsection{Přidány složky}
				Po stisknutí tlačítka \emph{Přidat složku} se opět zobrazí okno pro výběr souborů. Nyní je však možné vybrat pouze složky a žádné soubory nejsou zobrazeny. Po výběru složky probíhá akce identicky jako při přidáním jednotlivých souborů (to samé platí pro opětovné přidání).\\
				
				Opět platí, že pokud je okno zavřeno tlačítkem \emph{Storno} či křížkem, schová se a aplikace pokračuje v běhu bez jakékoliv chyby. 
				
			\subsubsection{Vybrání souboru}
				Ve chvíli, kdy jsou nějaké soubory v levém seznamu, je možné kliknout na název souboru (ne na zaškrtávací pole vedle), to soubor zvýrazní a také aktualizuje oblast \emph{File Details} v pravé dolní části. Zde by se měl zobrazit název daného souboru s kompletní cestou. Současně by se měly aktualizovat počty kontraktů dle typu. Také se zpřístupní tlačítko \emph{Show Details}.
				
			\subsubsection{Označení souborů}
				Soubory je možné označit pomocí zaškrtávacích polí nalevo od názvů souborů či pomocí tlačítka \emph{Select All}/\emph{Deselect All}. Jestliže nejsou označeny všechny soubory, tlačítko by mělo mít nápis \emph{Select All} a jeho stisknutí by mělo všechny soubory označit. Naopak pokud jsou všechny soubory označeny, měl by zde být nápis \emph{Deselect All}, což by mělo zrušit označení všech položek.\\
				
				Ať je použito tlačítko, či zaškrtávací políčka, měl by vždy se aktualizovat popisek pod seznamem oznamující počet označených souborů. Mimo to by se také měly vždy zpřístupnit tlačítka pro export a mazání souborů v horním panelu. Naopak pokud není žádný soubor označen, tlačítka by se měla opět zablokovat.
			
			\subsubsection{Mazání souborů}
				Tlačítko pro mazání souborů by mělo být přístupné pouze ve chvíli, kdy je označen alespoň jeden soubor. Po jeho stisku by měly ihned zmizet označené soubory ze seznamu a stejným způsobem jako při přidávání souborů by se měly aktualizovat všechny patřičné elementy rozhraní. Do konzole by se také měla napsat informace o počtu smazaných souborů. Pokud byly smazány všechny viditelné soubory, ale některé soubory zůstávají skryty, uživatele o tom informuje popisek zobrazený v seznamu. V opačném případě je zde napsáno, že list je prázdný a je možné soubory přidat pomocí tlačítek.
			
			\subsubsection{Export souborů}
				Stejně jako mazání souborů, je i export dostupný pouze v případě, pokud je označen alespoň jeden soubor. 
				
	
	\section{Testovací data}
		 - popis testovacích dat (syntetická, skutečná - výsledky testů)
		 
		 
	\section{Výsledky testů}
