\chapter{Zajištění kvality software}
	Jedním z obsáhlých odvětí softwarového inženýrství je zajištění kvality software. Mnoho institucí se touto problematikou zabývá a má velký význam jak pro komerční společnosti, tak pro výzkumné skupiny. V Této kapitole bude tato problematika stručně nastíněna a budou zde uvedeny různé možnosti zajištění kvality software. Obsah je čerpán zejména z článku Software Development Process and Software Quality Assurance \cite{swqa}.

\section{Kvalitní software}
	Aby bylo možné se bavit o možnostech zajištění kvality software, je třeba nejprve specifikovat, jaké vlastnosti určují, zda je daný software kvalitní. Klíčovou vlastností je samozřejmě správná funkčnost daného software nebo-li splnění funkčních požadavků. Mimo to je však na software kladena řada mimo-funkčních požadavků, jako je např. udržitelnost, stabilita, znovupoužitelnost atd. Důležitost dílčích vlastností je u každého projektu jiná a znalost jejich priority by měla být součástí správné analýzy.

	\subsection{Vlastnosti určující kvalitu software}
		Zde je seznam některých atributů, které určují kvalitu software:
	
		\subsubsection{Funkčnost}
			Je logické, že software musí splňovat požadovanou funkčnost, jinak by nebyl k prospěchu. V závislosti na typu projektu ale může být vhodné udělat kompromis za účelem zvýhodnění jiných vlastností.	
	
		\subsubsection{Udržitelnost}
			Určuje, jak obtížné je provést změny na daném software. Tyto změny mohou být za účelem oprav, přizpůsobení se novým požadavkům, přidání nové funkčnosti atp. Obecně je snahou, aby tyto změny bylo možné provádět s~využitím co nejmenšího množství zdrojů.
		
		\subsubsection{Spolehlivost}
			Spolehlivý systém by měl odolávat vnějším vlivům, jako jsou například výpadky či útoky, a neměl by způsobit škodu při selhání. V důsledku by pak měl být software co nejvíce dostupný.
		
		\subsubsection{Efektivita}
			Software by měl pracovat co nejefektivněji, tedy s co nejmenším využitím zdrojů. Často nás zajímá rychlost a nízké nároky na hardware.
		
		\subsubsection{Použitelnost}
			Kvalitní software by měl umožňovat snadné použití, což typicky bývá spjato s přátelským uživatelským rozhraním, ale může být ovlivněno i náročností instalace či spuštění.
		
		\subsubsection{Znovupoužitelnost}
			Při vývoji by se také mělo myslet na možnost znovu-použití již vytvořených komponent. Jednou vytvořené části se tak dají využití pro jiný projekt či jinou část aplikace, což omezuje duplicitu kódu a v důsledku šetří zdroje.
		
		\subsubsection{Testovatelnost}
			Dobrý software je možné kvalitně otestovat a je známá množina testovacích případů. Díky tomu lze lépe předcházet chybám.\\

Všechny tyto atributy určují kvalitu software a v závislosti na typu projektu by mělo být cílem každého týmu, dosáhnout co nejlepších výsledků v~daných oblastech. 


\section{Aspekty ovlivňující kvalitu software}
Po uvedení klíčových vlastností definující kvalitní systém je na místě prozkoumat možnosti zajištění těchto vlastností. Aspektů, které tyto vlastnosti ovlivňují je celá řada a zde je seznam některých z nich:

	\subsection{Metodika řízení softwarového projektu}
		Volba vhodné metodiky řízení projektu je velmi důležitá, protože ovlivní celý průběh vývoje. Tato volba je závislá na více faktorech jako je povaha a~rozsah projektu, velikost a zkušenosti týmu, který bude na projektu pracovat atd. V dnešní době se obecně dává přednost agilním metodikám jako je např. SCRUM, což platí zejména pro větší projekty.
	
	\subsection{Analýza požadavků}
		Analýza a sběr požadavků jsou jedny z prvních činností, které je třeba při tvorbě software provést. Jedná se o důležitý krok, jehož chyby se mohou posléze projevit v celém projektu a typicky mohou vést k vyšším nárokům na zdroje, což se může negativně odrazit na výsledné kvalitě software. Je třeba nalézt všechny aktéry a rozpoznat všechny případy užití. Na základě toho zpracovat funkční i mimo-funkční požadavky, které zákazník očekává a zároveň budou v kompetenci vývojářů. Důležité je také správně stanovit rozsah projektu a určit si hranice. 

	\subsection{Návrh systému}
		Na základě zpracovaných požadavků by měla být provedena analýza, která povede ke tvorbě několika kandidátních architektur, ze kterých by se nakonec měla zvolit architektura, která bude ve výsledku použita. Posléze může začít návrh systému na úrovni komponenty později tříd atd. V tomto kroku je důležité dbát na všechny funkční i mimo-funkční požadavky a vytvořit dostatečně robustní návrh, který se dokáže vyrovnat s menšími změnami.
		
	\subsection{Vývoj}
		Během vývoje je vhodné, aby vývojáři dbali na stanovené zásady programování v dané skupině. Cílem je, aby byl kód přehledný i pro ostatní členy týmu a aby byly snazší další potenciální úpravy. S tím souvisí komentování kódu a programování proti rozhraní, což značně zvyšuje znovupoužitelnost. Pro další zvýšení přehlednosti, vyhnutí se potenciálním chybám a zajištění splnění požadavků je také možné využít kontraktů softwarových rozhraní (viz níže).
	
	\subsection{Testování}
		Testování je z hlediska kvality důležitým aspektem celého projektu, protože může odhalit řadu chyb, které ji značně snižují. Může se jím předejít pádům systému, chybám ve funkčnosti, problémům s výkonem atd. Pro testování je třeba správná analýza testovacích případů a hraničních hodnot, aby bylo docíleno vysokého pokrytí.
	
	
\section{Preventivní techniky zajištění kvality}	
	Pro zajištění vysoké kvality software byla vytvořena řada preventivních metod, které umožňují předcházet chybám, v důsledku čehož vzniká kvalitnější software. V této části budou představeny některé z těchto technik.
		

	\subsection{Defenzivní programování}
		Defenzivní programování \cite{defProgram} je technika, ve které předpokládáme, že náš program obsahuje chyby a očekáváme nejhorší možný vstup. Myšlenkou tedy je, snažit se odhalit všechny potenciální problémy (jakkoliv drobné) a pokusit se jím předejít. Ve výsledku to pak znamená větší množství ověřování a kontrol než je zvyklé při běžném programování. Snahou programátora by také mělo být napsat co nejpřehlednější kód, který umožňuje znovupoužitelnost a snižuje šanci chyb.\\ 
		
		V rámci této techniky je důležité upravit své programovací návyky, které mohou vést k chybovosti kódu. Častou chybou může být např. ignorování návratového typu různých metod, které zpravují o její úspěšnosti. Problémy také mohou způsobovat neinicializované proměnné, u kterých nemáme žádnou informaci o tom, jakou obsahují hodnotu. Obecně je také potřeba vždy kontrolovat uživatelské vstupy a důkladně je ověřovat. Tím se můžeme vyhnout např. přístupu do jiné části paměti či práci s nulovými objekty. Pro zvýšení účinnosti kontrol můžeme také použít různé specializované techniky, které umožní lepší kontrolu našeho programu. Příkladem mohou být doporučení pro psaní bezpečného kódu či kontrakty (viz níže).\\
		
		Defenzivní programování je dobrý způsob, jak zvýšit přehlednost a snížit chybovost kódu, nicméně v extrémních případech můžeme docílit opačného efektu. Důvodem může být např. opakované kontrolování stejných hodnot, předcházení situacím, které nemohou nikdy nastat atd. Tuto techniku je tedy třeba brát s mírou.
		
	
	\subsection{Doporučení pro psaní kódu}
		Jednou z preventivních technik pro zajištění kvality jsou různá doporučení pro psaní kódu. Ta mohou být určena standardem, neoficiálními konvencemi či firemní politikou. Kód psaný na základě těchto doporučení bývá zpravidla více čitelný pro ostatní členy týmu, efektivnější a bezpečnější.\\
		
		Příkladem kódovacího standardu může být MISRA \cite{misra}. Jedná se o standard, který byl původně navržen pro zvýšení bezpečnosti a spolehlivosti vestavěných systémů v automobilech napsaných v jazyce C. Postupem času se rozšířil také do jazyka C++ a dostal se i mimo oblast automobilového průmyslu. 		
		
		\subsubsection{Analyzátory kódu}
			V rámci kontroly kódu je možné použít analyzátory kódu. Tyto nástroje kontrolují napsaný kód buď staticky při překladu či použitím daného nástroje. Tím vývojář získá rychlou odezvu k právě napsanému kódu a může jej ihned opravit. Nějaké analyzátory bývají zpravidla integrovány ve vývojovém prostředí a mnoho prostředí také umožňuje přidání jiných validátorů. Ověření probíhá na základě definované sady pravidel, která často bývá vytvořena na základě standardu či obecných \emph{best practises} (osvědčené postupy).
	
	\subsection{Kontrakty}
		Jednou z možností, jak zajistit kvalitu software, je také použití kontraktů. Ty reprezentují dohodu o použití dané komponenty, či metody, mezi vývojářem a uživatelem. Kladou určitá omezení na daný objekt, což snižuje šanci jeho špatného použití. Podrobnosti je možné nalézt v další kapitole, která je celá věnována kontraktům.