\chapter{Úvod}

S rozvojem objektově orientovaného programování se rozmohl trend dělení software do nezávislých komponent, které jsou snadno nahraditelné a lze je vyvíjet takřka nezávisle. Kromě mnoha nepopiratelných výhod této metodiky jsou zde samozřejmě také potenciální rizika. Jedním z možných rizik může být špatná komunikace těchto samostatných součástí. Sémantické kontrakty jsou jednou z možností, jak snížit chybovost těchto integrací a zvýšit jejich přehlednost. Z tohoto důvodu má smysl se těmito konstrukcemi zabývat a analyzovat jejich použití.\\

Jedním z cílů této diplomové práce je seznámit se s konceptem kontraktu softwarových modulů, zejména pak přístupem Design by Contract (DbC) a prostudovat způsoby popisu DbC kontraktu v Java technologiích. Primárním cílem je návrh a implementace nástroje pro extrakci, případně porovnání, konstrukcí DbC ze zdrojových, respektive přeložených, souborů jazyka Java. Součástí je také analýza a návrh modelu, který bude schopen takto získaná data reprezentovat. Závěrem práce bude ověření správnosti získaných výsledků a jejich souhrn.\\ 

Po přečtení této práce by měl čtenář získat základní informace o tom, co to jsou kontrakty, jakým způsobem se rozdělují a jaký mají vliv na kvalitu software. Podrobněji by se měl dozvědět o design by contract a různých způsobech jeho reprezentace. Čtenář také bude uveden do problematiky rozboru zdrojových i přeložených souborů jazyka Java a zejména pak s možnostmi extrakce kontraktů z těchto dat. V druhé části práci získá čtenář informace o implementací daného nástroje a jakým způsobem jsou v něm reprezentována data. Závěrem se dozví podrobnosti o testování a dosažených výsledcích.