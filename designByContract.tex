\chapter{Popis kontraktů softwarových rozhraní}

%%%%%%%%%%%%%%%%%%%%%%%%%%%%%%%%%%%%%%%%%%%%%%%%%%%%%%%%%%%%%%%%%%%%%%%%%%%%%%%%%%%%%%%%%%%%%%%%%%%%%%%%%%%%%%%%%%%%%%%%%%%%%%%%%%%%%%%%%%%%%%%%%%%%%%%%%%%%%%%%%%%%%%%%%%%%%%%%%%%%%%%%%%%%%%%%%%%%%%
%%%%%%%%%%%%%%%%%%%%%%%%%%%%%%%%%%%%%%%%%%%%%%%%%%%%%%%%%%%%%%%%%%%%%%%%%%%%%%%%%%%%%%%%%%%%%%%%%%%%%%%%%%%%%%%%%%%%%%%%%%%%%%%%%%%%%%%%%%%%%%%%%%%%%%%%%%%%%%%%%%%%%%%%%%%%%%%%%%%%%%%%%%%%%%%%%%%%%%
	\section{Koncept kontraktů softwarových modulů}	
		Abychom v softwarovém inženýrství zajistili znovupoužitelnost a bezchybnost nezávislých komponent, je třeba specifikovat, jakým způsobem se mají používat a jak s nimi komunikovat. Jedná se o kontrakt mezi tím, kdo komponentu implementoval (dodavatel, vývojář) a tím, kdo ji používá (klient, uživatel). Vývojář zaručuje, že modul bude fungovat dle specifikace, za předpokladu, že bude používán správně. Text této kapitoly čerpá primárně z těchto zdrojů: \cite{contractsInWild}\cite{applyingDbc}\cite{ooswConstruction}\cite{contractAware}.\\		
		
		V této kapitole bude čtenář kromě konceptu kontraktů také seznámen s vlivem použití na kvalitu kódu a bude zde rozebrán koncept design by contract. Následovat bude rozdělení kontraktů design by contract a příklady nástrojů pro práci s kontrakty pro jazyk Java, ale i jiné technologie.\\
		
		\subsection{Úrovně kontraktů}		
			Kontrakty je možné dělit do čtyř úrovní, dle toho, jak jsou otevřené diskuzi, kde první úroveň je neměnná a čtvrtá je dynamická a otevřená změnám: 
			\begin{itemize}
				\item 1. úroveň - syntaktické
				\item 2. úroveň - sémantické
				\item 3. úroveň - interakční
				\item 4. úroveň - mimo-funkční
			\end{itemize}
		
		\subsubsection{Syntaktické kontrakty}
			Základní vrstvou kontraktů jsou kontrakty syntaktické. Jejich znění je neměnné a jedná se o nutnou podmínku pro dodržení dohody mezi vývojářem a uživatelem. Specifikují operace, které může daná komponenta provádět, vstupní a výstupní parametry komponenty a výjimky, které během daných operací mohou nastat. Můžeme tedy říci, že pokrývají signatury a definice rozhraní použitých konstrukcí. 
		
		\subsubsection{Sémantické kontrakty}
			Úroveň sémantických kontraktů specifikuje chování definovaných operací, což umožňuje zabránit jejich chybnému použití a také zvyšuje přehlednost a transparentnost daného rozhraní. Vytyčuje hraniční hodnoty za použití operací \emph{assert}\footnote{Operace porovnání, která porovná reálnou hodnotu s hodnotou očekávanou. Pokud se tyto hodnoty neshodují nastane výjimka. Tato operace bývá často spojována s testováním.} (dále aserce), respektive pomocí definic \emph{pre-conditions} (dále vstupní podmínky), \emph{post-conditions} (dále výstupní podmínky) a \emph{class invariants} (dále neměnné podmínky). Vstupní podmínky kladou požadavky na vstupní argumenty, kontrolují se tedy na začátku operace. Výstupní podmínky specifikují omezení pro výstup operace a jsou tedy vyhodnoceny po dokončení operace. Neměnné podmínky kladou požadavky na vstup i výstup každé veřejné operace v dané třídě. Trojice těchto podmínek využívá aserce a je součástí konceptu design by contract, kterému je věnována část práce níže. Zde je příklad v pseudokódu znázorňující princip sémantických kontraktů:\\\\
			\- \- \- \- \- \texttt{method number example(number x)\{}\\
			\- \- \- \- \- \- \- \- \- \- \texttt{\textcolor{pgrey}{// Vstupní podmínka na parametr x}}\\ 
        	\- \- \- \- \- \- \- \- \- \- \texttt{require(x > 0, "x has to be a positive number")}\\\\
        	\- \- \- \- \- \- \- \- \- \- \texttt{...}\\\\ 
        	\- \- \- \- \- \- \- \- \- \- \texttt{\textcolor{pgrey}{// Výstupní podmínka na vrácení proměnné x}}\\ 
        	\- \- \- \- \- \- \- \- \- \- \texttt{ensure(x < 100, "x has to be lesser than 100")}\\\\
			\- \- \- \- \- \- \- \- \- \- \texttt{return x}\\ 
    		\- \- \- \- \- \texttt{\}}\\
    		
			V příkladu je znázorněna metoda se vstupem v podobě čísla \texttt{x}. Je zde vstupní podmínka, která říká, že \texttt{x} musí být větší než 0. Pokud bude tato podmínka porušena, nastane výjimka a vypíše se zpráva \texttt{"x has to be a positive number"}. Obdobným způsobem funguje i výstupní podmínka, která je vyhodnocena před návratem z metody.
		
		\subsubsection{Interakční kontrakty}
			Definují chování operací komponenty na úrovni synchronizace. Předchozí úrovně kontraktů považují jednotlivé operace za atomické, což samozřejmě nemusí být vždy pravda. Tato vrstva specifikuje paralelismy komponenty a s tím spjaté synchronizační prostředky.
		
		\subsubsection{Mimo-funkční kontrakty}
			Tyto kontrakty určují mimo-funkční požadavky na danou komponentu. Typicky se jedná o různé vlastnosti, které zlepšují kvalitu dané služby. Může to být např. doba odezvy, přesnost výsledku apod. (viz Kapitola 2. Zajištění kvality software).
		
	\subsection{Vliv na kvalitu kódu a software}
		Použití kontraktů v kódu přináší mnoho výhod, které mohou zvýšit kvalitu vývoje, respektive pak výsledného softwaru. Často vynucují správné chování při statické nebo dynamické kontrole a zajišťují tak správnost toku dat. Poskytují dodatečné informace při popisu rozhraní a pomáhají tak v lepší orientaci v projektu. Při použití kontraktů tak vývojář ví, jaké nároky může mít na danou operaci, a co se na oplátku očekává, že dodrží. Použití kontraktů může také pomoci při debuggingu, či při analýze vstupů a výstupů.\\
		
		Z využití kontraktů však mohou také plynout určité nevýhody. Jednou z nich je chybné použití kontraktů důsledkem špatné analýzy, které může vést k různým problémům. Kontrakt  může být příliš omezující a bránit tak plnému využití funkce, či naopak může být příliš volný a dovolovat nevalidní hodnoty. V závislosti na typu daného kontraktu může také dojít ke zvýšení režie a tedy zpomalení vykonávaného kódu, což by mohl být problém zejména u časově kritických operací. Obecně ale platí, že při zodpovědném používání, mohou být kontrakty velice prospěšné a přispět ke zlepšení kvality vyvíjeného software.


%%%%%%%%%%%%%%%%%%%%%%%%%%%%%%%%%%%%%%%%%%%%%%%%%%%%%%%%%%%%%%%%%%%%%%%%%%%%%%%%%%%%%%%%%%%%%%%%%%%%%%%%%%%%%%%%%%%%%%%%%%%%%%%%%%%%%%%%%%%%%%%%%%%%%%%%%%%%%%%%%%%%%%%%%%%%%%%%%%%%%%%%%%%%%%%%%%%%%%
%%%%%%%%%%%%%%%%%%%%%%%%%%%%%%%%%%%%%%%%%%%%%%%%%%%%%%%%%%%%%%%%%%%%%%%%%%%%%%%%%%%%%%%%%%%%%%%%%%%%%%%%%%%%%%%%%%%%%%%%%%%%%%%%%%%%%%%%%%%%%%%%%%%%%%%%%%%%%%%%%%%%%%%%%%%%%%%%%%%%%%%%%%%%%%%%%%%%%%	
	\section{Design by contract}
		Pojem design by contract zavedl francouzský profesor Bertrand Meyer \cite{meyerBio}\cite{eiffelStudio}. První větší zmínka je uveden v publikaci \emph{Design by Contract, Technical Report} v roce 1987. B. Meyer v průběhu let působil na řadě univerzit jako např. v Politecnico di Milano či ETH Zurich a je autorem mnoha publikací a knih. Mimo design by contract, byl jeho významným příspěvkem do oblasti softwarového inženýrství programovací jazyk Eiffel, který je s DbC úzce spjat.\\	
	
		Hlavním cílem design by contract je zvýšení spolehlivosti a správnosti u rozsáhlých softwarových projektů. Principem DbC je zajištění formální dohody mezi vývojářem a uživatelem určitého softwarového modulu. Jak bylo zmíněno výše, DbC je spjato se třemi typy podmínek (vstupní podmínky, výstupní podmínky a neměnné podmínky). O neměnných podmínkách je také možné říci, že se jedná o vstupní a zároveň výstupní podmínky vše veřejných metod v dané třídě. Není nutné aby tyto podmínky platily v průběhu jednotlivých operací.\\
	
	Podmínky jsou definovány pomocí konstrukcí v kódu programu. V závislosti na typu daného kontraktu, mohou poskytovat statickou kontrolu a/nebo jsou ověřovány při běhu. V případě, že byla některá z nich porušena, je vyvolána výjimka. Tímto chováním je zajištěno, že kontrakt bude dodržen. I přesto, že sémantické kontrakty mohou působit dojmem, že slouží jako náhrada testů, nejedná se o zaměnitelné funkce a naopak by se měly navzájem doplňovat.	
	
	
	\section{Rozdělení sémantických kontraktů}
		Kontrakty můžeme rozdělit do několika kategorií dle způsobu jejich použití:
			
		\begin{itemize}
			\item Podmíněné výjimky za běhu (Conditional Runtime Exceptions - CRE)
			\item API (využití metod knihovny)
			\item Assert (použití příkazu \texttt{assert})
			\item Anotace (specifikace kontraktů pomocí anotací)
			\item Ostatní
		\end{itemize}				
		
		\paragraph{CRE}
		 Nejběžnějším způsobem pro specifikaci kontraktů jsou podmíněné výjimky, které jsou vyvolány za běhu při porušení kontraktu (podmínky). K dispozici jsou různé typy výjimek, které je možné použít, mezi ně patří např. IllegalStateException, IllegalArgumentException, NullpointerException, IndexOutOfBoundsException či UnsupportedOperationException. Tyto nebo analogické výjimky jsou součástí většiny dnes běžných programovacích jazyků a prostředí, což je jeden z faktorů, proč je tento způsob tak četný.
				
		\paragraph{API}
		Další možností implementace kontraktů je využití specializovaného API, které poskytuje metody pro práci s kontrakty. Typicky se jedná o rozšířenou práci s výjimkami, se kterou se navenek pracuje jako se statickými metodami. Tato API zpravidla poskytují širší možnosti a umožňují tak sofistikovanější práci s kontrakty.
		
		\paragraph{Assert}
		Použitím klíčového slova \texttt{assert} je také možné kontrakty vytvářet. Stejně jako v případě výjimek, i zde se jedná o standardní součást jazyka. Aserce je tvrzení o stavu programu, které vyvolá výjimku není-li dodrženo. Aserce je typicky spojována s tvorbou testů, ale je možné ji použít i pro definici kontraktů.
		
		\paragraph{Anotace}
		Dalším způsobem je využití anotací, pomocí kterých je také možné specifikovat kontrakty. Anotace je možné uvádět před specifikaci tříd či metod nebo například i před parametry, v závislosti na dané anotaci. Některé anotace pro specifikaci kontraktů poskytuje také standardní knihovny Java, nicméně pro pokročilejší funkce je třeba využít externí zdroje.
		
		\paragraph{Ostatní}
		Existují také různé specifické způsoby definice kontraktů, které nepatří do žádné z těchto čtyř kategorií, nicméně nejsou příliš časté. Příkladem této kategorie může být jContractor (viz níže).
	


%%%%%%%%%%%%%%%%%%%%%%%%%%%%%%%%%%%%%%%%%%%%%%%%%%%%%%%%%%%%%%%%%%%%%%%%%%%%%%%%%%%%%%%%%%%%%%%%%%%%%%%%%%%%%%%%%%%%%%%%%%%%%%%%%%%%%%%%%%%%%%%%%%%%%%%%%%%%%%%%%%%%%%%%%%%%%%%%%%%%%%%%%%%%%%%%%%%%%%
%%%%%%%%%%%%%%%%%%%%%%%%%%%%%%%%%%%%%%%%%%%%%%%%%%%%%%%%%%%%%%%%%%%%%%%%%%%%%%%%%%%%%%%%%%%%%%%%%%%%%%%%%%%%%%%%%%%%%%%%%%%%%%%%%%%%%%%%%%%%%%%%%%%%%%%%%%%%%%%%%%%%%%%%%%%%%%%%%%%%%%%%%%%%%%%%%%%%%%	
	\section{Technologie pro popis sémantických kontraktů v jazyce Java}
		V Java existuje celá řada nástrojů, které umožňují práci s kontrakty. Liší se v nabízených možnostech a ve způsobech, jak se s nimi pracuje. Některé z nich se již dále nevyvíjejí nicméně jsou stále používány. V tomto oddíle podrobněji rozebereme některé z nich.		
	
		\subsection{Guava Preconditions}
			Knihovna Guava \cite{guava} od Google poskytuje řadu nových funkcí jako například různé kolekce, primitiva, práce se souběžnými programy atd. Z hlediska kontraktů je pro nás však zajímavá pouze třída \texttt{Preconditions}, která poskytuje metody pro validaci různých stavů. Je zde řada metod, které typicky začínají klíčovým slovem \texttt{check*} (např. \texttt{checkArgument}, \texttt{checkState}, \texttt{checkNotNull} atd.). Tyto metody jsou použity běžně v kódu programu a poskytují kontrolu pro vstupní argumenty, jedná se tedy pouze o definice vstupních podmínek, jak již název napovídá. Při porušení takovéto podmínky je pak vyvolána výjimka při běhu programu. Volitelným parametrem každé metody je také zpráva, která má být při porušení kontraktu zobrazena. Tuto zprávu je pak možné parametrizovat dalšími argumenty.\\
			
			Guava Preconditions poskytuje dobré prostředí pro práci s kontrakty, avšak její nevýhodou je, že je omezena pouze na vstupní podmínky. Zde je vidět příklad použití Preconditons v kódu:\\\\
			\- \- \- \- \- \texttt{public void guavaPreconditionsExample(Object x)\{}\\
			\- \- \- \- \- \- \- \- \- \- \texttt{String message = "x cannot be null.";}\\ 
        	\- \- \- \- \- \- \- \- \- \- \texttt{Preconditions.checkNotNull(x, message);}\\
    		\- \- \- \- \- \texttt{\}}\\
    		
    		V tomto příkladu je vstupní parametr \texttt{Object x} omezen a vstupem nemůže být hodnota \texttt{null}, pokud se tak stane, je vyhozena standardní výjimka \texttt{java.lang.NullPointerException}. Typ výjimky, který knihovna vrací je závislý na dané metodě (např. v případě metody \texttt{checkArgument(boolean)} se jedná výjimku \texttt{IllegalArgumentException}).\\
    		
    		Tento nástroj je hojně používány, což lze vyvodit z textu Contracts in Wild \cite{contractsInWild}, ale i z množství zmínek a návodů na internetu (např. \cite{guavaTutorial}). Jedná se také o živý projekt, který je stále aktualizován.
		
		
		\subsection{JSR305}
			JSR305 \cite{jsr305} umožňuje specifikaci kontraktů pomocí anotací. Na rozdíl od Guava Preconditions umožňuje také specifikaci výstupních i neměnných podmínek. Obecně platí, že anotace, které jsou uvedeny před argumenty metod např: \texttt{@Nonnull Object x} specifikují vstupní podmínku pro parametr dané metody. Pokud je anotace uvedena pro celou metodu, jedná se o výstupní podmínku a kontrakt se tak váže na výstupní hodnotu metody. V případě, že je anotace vázaná na třídu, jedná se o neměnnou podmínku. Některé anotace je možné použít jako libovolný druh, nicméně mnoho z nich je specializována pro jeden či dva typy podmínek.\\ 
			
			Pokud je kontrakt porušen, je opět vyhozena výjimka, v tomto případě \texttt{IllegalArgumentException}. Na rozdíl od Guava, zde není nativní možnost pro zadání vlastní chybové zprávy v případě porušení kontraktu, ale výchozí chyba je poměrně samovysvětlující. JSR305 nicméně neposkytuje pouze striktní podmínky, které při porušení skončí chybou, ale umožňuje také anotace, které slouží jako informace pro vývojáře. Např. anotace \texttt{@CheckForNull} upozorňuje, že daný objekt může nabýt hodnoty \texttt{null}, ale nevynucuje žádné chování a je pouze na vývojáři, jak s touto informací naloží. Příklad zobrazující všechny tři typy kontraktů je vidět zde:\\\\			
			\- \- \- \- \- \texttt{@ParametersAreNullableByDefault}\\
			\- \- \- \- \- \texttt{public class JSR3053ExampleClass \{}\\\\
   			\- \- \- \- \- \- \- \- \- \- \texttt{@CheckReturnValue}\\
    		\- \- \- \- \- \- \- \- \- \- \texttt{public Object JSR305Example(@Nonnull Object x)\{}\\
      		\- \- \- \- \- \- \- \- \- \- \- \- \- \- \- \texttt{\textcolor{pgrey}{// other code}}\\
    		\- \- \- \- \- \- \- \- \- \- \texttt{\}}\\
			\- \- \- \- \- \texttt{\}}\\
			
			JSR značí Java Specification Requests, tedy specifikační požadavky pro Java. Jedná se o popisy finálních specifikací pro jazyk Java. Jednotlivé JSR se postupně schvalují a zhodnocují a jejich průběžný stav je možné sledovat. JSR305 rozšiřuje standardní knihovnu \texttt{javax.annotations}. I přesto, že JSR305 je ve stavu \emph{dormant}\footnote{\emph{Dormant} značí, že práce na tomto JSR projektu byla pozastavena. Může to být na základě hlasování komise, či protože dané JSR dosáhlo konec své životnosti.}, stále je hojně využíváno v řadě projektů a má tak smysl jej zkoumat.
			
	
		\subsection{Cofoja}
			Contracts for Java \cite{cofoja}, či zkráceně Cofoja, je aplikační rámec, který mimo jiné umožňuje práci s kontrakty. Kontrakty jsou definovány na úrovni anotace a slouží pouze ke kontrole za běhu programu, neposkytují tedy statickou kontrolu. Cofoja umožňuje použití všech tří typů podmínek a zajišťuje to pomocí klíčových slov \texttt{@Requires} pro vstupní podmínky, 
\texttt{@Ensures} pro výstupní podmínky a \texttt{@Invariant} pro neměnné podmínky. Praktické využití v kódu pak může vypadat takto:\\\\
				\- \- \- \- \- \texttt{@Requires("x >= 0")}\\
				\- \- \- \- \- \texttt{@Ensures("result >= 0")}\\
				\- \- \- \- \- \texttt{static double sqrt(double x);}\\
				
			Nejnovější verzí tohoto nástroje je 1.3, která byla vydána v únoru 2016. Poslední aktivita proběhla v srpnu 2017, je tak možné se domnívat, že se v současné době jedná o definitivní verzi.

		\subsection{valid4j}
			Valid4j \cite{valid4j} je jednoduchý nástroj, který poskytuje metody pro práci s kontrakty za pomoci aserce. Podobně jako jiné nástroje využívá klíčových slov \texttt{require} pro vstupní a \texttt{ensure} pro výstupní podmínky. Tento nástroj neposkytuje podporu pro neměnné podmínky za pomocí specializovaných metod ale tohoto chování je možné dosáhnout použitím zmíněných metod. Zde je část kódu implementující kontrakty nástrojem valid4j.\\\\  
				\- \- \- \- \- \texttt{public Stuff method(Object param) \{}\\
				\- \- \- \- \- \- \- \- \- \- \texttt{require(param, notNullValue());}\\
				\- \- \- \- \- \- \- \- \- \- \texttt{require(getState(), equalTo(GOOD));}\\
				\- \- \- \- \- \- \- \- \- \- \texttt{...}\\
				\- \- \- \- \- \- \- \- \- \- \texttt{ensure(getState(), equalTo(GREAT));}\\
				\- \- \- \- \- \- \- \- \- \- \texttt{return ensure(r, notNullValue());}\\
				\- \- \- \- \- \texttt{\}}\\	
				
			Poslední verzí tohoto nástroje je 0.5.0 z prosince 2015. Krátce poté se také zastavila aktivita v repozitáři GitHub, můžeme tak předpokládat, že projekt byl prozatím ukončen.
	
		
		\subsection{jContractor}
			Posledním příkladem je jContractor \cite{jcontractor}. Z hlediska definice kontraktů se jedná se o unikátní nástroj, který nepatří do žádné ze zmíněných skupin. Kontrakty jsou zde definovány tvorbou metod s danou jmennou konvencí. To znamená, že pokud chceme vytvořit vstupní podmínky pro metodu \texttt{push}, musíme definovat metodu \texttt{push\_Precondition}. Obdobně to platí i pro výstupní podmínky s příponou \texttt{\_Postcondition} a neměnné podmínky s příponou \texttt{\_Invariant}. Jedná se o metody, jejichž návratovým typem je \texttt{boolean}, který určuje, zda byla podmínka splněna, či nikoliv. Podmínky jsou pak do kódu přidány při generování \emph{bytecode}\footnote{Programy psané v jazyce Java nejsou překládány přímo do strojového kódu, jak je typické, ale jsou překládány do tzv. bajtkódu, což umožňuje platformní nezávislost. Více informací je popsáno níže, ve 4. kapitole Tokenizace jazyka Java} (dále bajtkód) a zajišťují tak kontrolu pouze při běhu programu. Zde je krátká ukázka kódu demonstrující použití nástroje jContractor:\\\\  
				\- \- \- \- \- \texttt{\textcolor{pgrey}{// metoda push s vlastním kódem}}\\
				\- \- \- \- \- \texttt{public void push (Object o) \{}\\
				\- \- \- \- \- \- \- \- \- \- \texttt{...}\\
				\- \- \- \- \- \texttt{\}}\\\\
				\- \- \- \- \- \texttt{\textcolor{pgrey}{// metoda definující vstupní podmínku pro push}}\\
				\- \- \- \- \- \texttt{protected boolean push\_Precondition (Object o) \{}\\
				\- \- \- \- \- \- \- \- \- \- \texttt{return o != null;}\\
				\- \- \- \- \- \texttt{\}}\\
				

%%%%%%%%%%%%%%%%%%%%%%%%%%%%%%%%%%%%%%%%%%%%%%%%%%%%%%%%%%%%%%%%%%%%%%%%%%%%%%%%%%%%%%%%%%%%%%%%%%%%%%%%%%%%%%%%%%%%%%%%%%%%%%%%%%%%%%%%%%%%%%%%%%%%%%%%%%%%%%%%%%%%%%%%%%%%%%%%%%%%%%%%%%%%%%%%%%%
%%%%%%%%%%%%%%%%%%%%%%%%%%%%%%%%%%%%%%%%%%%%%%%%%%%%%%%%%%%%%%%%%%%%%%%%%%%%%%%%%%%%%%%%%%%%%%%%%%%%%%%%%%%%%%%%%%%%%%%%%%%%%%%%%%%%%%%%%%%%%%%%%%%%%%%%%%%%%%%%%%%%%%%%%%%%%%%%%%%%%%%%%%%%%%%%%%%	
	\section{Popis sémantických kontraktů v jiných programovacích jazycích}
		Použití kontraktů samozřejmě není omezeno pouze na Java, ale je rozšířeno do mnoha jiných jazyků. Mimo použití běžně dostupných prostředků, jako jsou například výjimky či aserce, které jsou k dispozici téměř v každém jazyce, existují také specializované nástroje, které umožňují rozšířenou práci s kontrakty. Následuje krátký výčet některých z nich.
		
		\subsection{Code Contracts v .NET}
			Prvním příkladem může být projekt Code Contracts \cite{codeContracts}\cite{codeContracts2}, který byl vyvinut společností Microsoft a umožňuje použití kontraktů v .NET jazycích. Jedná se o open-source knihovnu, která formou API poskytuje funkce pro specifikaci kontraktů. Funguje na jednoduchém princip volání funkcí podobně jako Guava Preconditions, nicméně umožňuje použití všech tří typů podmínek. Základními funkcemi jsou \texttt{Contract.Requires()} pro definici vstupních podmínek, \texttt{Contract.Ensures()} pro zajištění správného výstupu a \texttt{Contract.Invariant()} pro reprezentaci neměnných podmínek. Mimo těchto základních funkcí poskytuje také různé specifické operace, se kterými je možné vytvářet komplexnější kontrakty. Umožňuje také např. specifikaci vyhozené výjimky. V následujícím příkladu je vidět použití základních konstrukcí:\\\\
			\- \- \- \- \- \texttt{Contract.Requires( x != null );}\\
			\- \- \- \- \- \texttt{Contract.Ensures( this.F > 0 );}\\
			\- \- \- \- \- \texttt{Contract.Invariant(this.y >= 0);}

			
		\subsection{PhpDeal v PHP}
			Jedním z nástrojů, které umožňují reprezentaci kontraktů pro jazyk PHP je aplikační rámec PhpDeal \cite{phpDeal}. Funguje na principu klíčových slov v dokumentačních komentářích, ty jako argument obsahují řetězec s PHP kódem, pomocí kterého je možné konstruovat podmínky. Při definici vstupních a výstupních podmínek jsou konstrukce uvedeny u dané funkce, neměnné podmínky jsou pak definovány v rámci třídy. Kontrakty jsou definovány klíčovým slovem \texttt{@Contract}, kde za zpětným lomítkem následuje \texttt{Verify} pro vstupní podmínku, \texttt{Ensure} pro výstupní podmínku a \texttt{Invariant} pro neměnnou podmínku. V závorce pak následuje řetězec s podmínkou, která je ověřována. Podmínky jsou vyhodnocovány při běhu programu v závislosti na definici nástroje. Je doporučeno vypnout vyhodnocování pro produkční verzi. Nástroj také umožňuje integraci do IDE pro zvýraznění syntaxe. Použití je vidět na následujícím příkladu:\\\\
			\- \- \- \- \- \texttt{/** @Contract\textbackslash Verify("\$amount\textgreater 0 \&\& is\_numeric(\$amount)")}\\	
			\- \- \- \- \- \texttt{ * @Contract\textbackslash Ensure("\$this-\textgreater bal == \$\_\_old-\textgreater bal+\$amount")}\\
			\- \- \- \- \- \texttt{ */}\\
			\- \- \- \- \- \texttt{public function deposit(\$amount)}\\
			\- \- \- \- \- \texttt{\{ ... \}} \\
			
			PhpDeal není příliš rozšířený nástroj, ale je snadný na použití. Poslední aktualizace proběhla na konci roku 2017, 
		
		\subsection{Boost.Contract v C++}
			Knihovna Boost.Contract \cite{boostContract} poskytuje podporu pro design by contract v jazyce C++. Pracuje na principu API a poskytuje tak funkce pro realizaci podmínek. Funkce se nazývají \texttt{precondition} a \texttt{postcondition}. Pro realizaci neměnných proměnných slouží definice funkce jménem \texttt{invariant}, která pak zajišťuje kontrolu všech vstupů a výstupů. Pro aserci je používána funkce \texttt{BOOST\_CONTRACT\_ASSERT}.
			
		
		\subsection{Jazyk Eiffel}
			Některé programovací jazyky konstrukce DbC podporují nativně a není tak třeba používat žádné dodatečné nástroje. Jedním z předních zástupců této kategorie je jazyk Eiffel \cite{eiffel}, který byl vyvíjen s cílem zajistit co největší spolehlivost v rámci programovacího jazyka. Jak již bylo zmíněno výše, byl navržen Bertrandem Meyerem, tedy stejným člověkem, který představil design by contract. Jazyk poskytuje mnoho zajímavých konstrukcí, zde bude však nastíněno použití konstrukcí DbC.\\
			
			 Definice je zajištěna klíčovými slovy \texttt{require} pro vstupní podmínky, \texttt{ensure} pro výstupní podmínky a \texttt{invariant} pro neměnné podmínky. Vzhledem k povaze jazyka jsou vstupní a výstupní podmínky začleněny přímo do těla funkce, které jsou zde však nazývány \texttt{feature}. Neměnné podmínky jsou pak samostatně definovány v bloku \texttt{invariant}. Eiffel poskytuje různé rozšířené možnosti pro práci s kontrakty, základní principy by měly být patrné z následujícího příkladu:\\\\
			\- \- \- \- \- \texttt{feature}\\
			\- \- \- \- \- \- \- \- \- \- \texttt{deposit (sum: INTEGER)}\\
			\- \- \- \- \- \- \- \- \- \- \- \- \- \- \- \texttt{require}\\
			\- \- \- \- \- \- \- \- \- \- \- \- \- \- \- \- \- \- \- \- \texttt{non\_negative: sum \textgreater = 0}\\
			\- \- \- \- \- \- \- \- \- \- \- \- \- \- \- \texttt{do}\\
			\- \- \- \- \- \- \- \- \- \- \- \- \- \- \- \- \- \- \- \- \texttt{...}\\
			\- \- \- \- \- \- \- \- \- \- \- \- \- \- \- \texttt{ensure}\\
			\- \- \- \- \- \- \- \- \- \- \- \- \- \- \- \- \- \- \- \- \texttt{updated: bal = old bal + sum}

%%%%%%%%%%%%%%%%%%%%%%%%%%%%%%%%%%%%%%%%%%%%%%%%%%%%%%%%%%%%%%%%%%%%%%%%%%%%%%%%%%%%%%%%%%%%%%%%%%%%%%%%%%%%%%%%%%%%%%%%%%%%%%%%%%%%%%%%%%%%%%%%%%%%%%%%%%%%%%%%%%%%%%%%%%%%%%%%%%%%%%%%%%%%%%%%%%%%%%
%%%%%%%%%%%%%%%%%%%%%%%%%%%%%%%%%%%%%%%%%%%%%%%%%%%%%%%%%%%%%%%%%%%%%%%%%%%%%%%%%%%%%%%%%%%%%%%%%%%%%%%%%%%%%%%%%%%%%%%%%%%%%%%%%%%%%%%%%%%%%%%%%%%%%%%%%%%%%%%%%%%%%%%%%%%%%%%%%%%%%%%%%%%%%%%%%%%%%%	
	\section{Souhrn technologií pro popis kontraktů}
		Výčet nástrojů výše je pouze omezený výběr příkladů, nikoliv kompletní seznam. To poukazuje na skutečnost, že nástrojů, které poskytují možnost práce se sémantickými kontrakty, existuje mnoho. Často se jedná o knihovny, které nabízejí širší možnosti použití než samotné kontrakty. V tabulce \ref{table:tabContracts} je vidět souhrn uvedených nástrojů. 		
		
		\begin{table}[H]
		\catcode`\-=12
		\begin{tabular}{|c||c|c|c|c|c|}
			\hline
			\multirow{2}{*}{\bf Název} & \multirow{2}{*}{\bf Jazyk} & \multirow{2}{*}{\bf Typ} & \multicolumn{3}{c|}{\bf Dostupné podmínky}\\
				\cline{4-6}
					& & & {\bf Vstup.} & {\bf Výstup.} & {\bf Neměn.}\\
				\hline
				\hline
					  Guava & Java & API & ANO & NE & NE\\
					  JSR305 & Java & Anotace & ANO & ANO & ANO\\
					  Cofoja & Java & Anotace & ANO & ANO & ANO\\
					  valid4j & Java & API & ANO & ANO & NE*\\
					  jContractor & Java & Speciální & ANO & ANO & ANO\\
					  Code Contracts & .NET & API & ANO & ANO & ANO\\
					  PhpDeal & PHP & Anotace** & ANO & ANO & ANO\\					  
					  Boost.Contract & C++ & API & ANO & ANO & ANO\\
				\hline
		\end{tabular}\\
				
		\footnotesize
			\noindent
			\- \- \- \textbf{*} Neměnné podmínky je možné vytvořit pomocí vstupních a výstupních podmínek.
			\- \- \- \- \- \- \- \textbf{**} Anotace jsou následovány funkčním PHP kódem
					
		\caption{Souhrn nástrojů pro práci s kontrakty}
		\label{table:tabContracts}
		\end{table}		
		
		\normalsize	
			
		Co se základní funkcionality týče, až na výjimky nástroje poskytují srovnatelné možnosti. Typově se pak obvykle jedná o nástroje, které zprostředkovávají metody pomocí API či o anotace. Obecně se dá říci, že volba nástroje záleží spíše na preferencích skupiny, která bude daný nástroj používat. Je možné se rozhodovat podle toho, jaký typ reprezentace daný nástroj používá, či jak je technologie známá a oblíbená. Z tohoto důvodu má smysl zabývat se problematikou, jak zajistit unifikaci specifikací kontraktů z různých nástrojů a poskytnout tak jednotnou reprezentaci, která umožní další analýzu. 