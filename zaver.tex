\chapter{Závěr}	
	Cílem práce bylo se seznámit s konceptem kontraktu softwarových modulů, zejména pak přístupem Design by Contract (DbC) a prostudovat způsoby popisu DbC kontraktu v Java technologiích. Primárním cílem bylo navrhnout a implementovat nástroj pro extrakci, případně porovnání, konstrukcí DbC ze zdrojových, respektive přeložených, souborů jazyka Java. Zároveň měla být provedena analýza a návrh modelu, který bude schopen takto získaná data reprezentovat. Závěrem bylo za úkol ověřit správnost získaných výsledků a jejich souhrn.\\
	
	V rámci teoretického úvodu byly popsány možnosti zajištění kvality software, na což bylo navázáno kapitolou věnující se kontraktům softwarových rozhraní. Zde jsem popsal, jaké druhy kontraktů rozlišujeme, podrobnosti o Design by Contract a té informace a souhrn různých druhů specifikace kontraktů. Součástí teoretické části byl také popis jazyka Java z hlediska gramatiky a kompilace. Následně byly také uvedeny informace o analýze a dekompilaci jazyka.\\
	
	Dále je popsána analýza a návrh modelu pro reprezentaci extrahovaných kontraktů. Další částí je popis implementace nástroje z hlediska knihovny i uživatelské aplikace. Na to navazuje část popisující testování obou částí a také zhodnocení výsledků.\\
	
	Nástroj byl úspěšně implementován a umožňuje extrakci konstrukcí kontraktů Guava Preconditions a JSR305. Nástroj také umožňuje porovnání těchto konstrukcí i z hlediska souborů a celých adresářů. Nástroj je možné ovládat pomocí vytvořené uživatelské aplikace, která umožňuje použití příkazů pro rychlé zpracování většího množství souborů či využití grafického rozhraní pro přehlednou práci.\\
	
	Primárního cíle práce tedy bylo úspěšně dosaženo stejně tak, jako všech ostatních cílů. Práci je tedy možné použít pro analýzu konstrukcí DbC a dále jí snadno rozšiřovat dle potřeby výzkumu.