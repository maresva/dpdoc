\chapter{Závěr}	
	Cílem práce bylo se seznámit s konceptem kontraktu softwarových modulů, zejména pak přístupem Design by Contract (DbC) a prostudovat způsoby popisu DbC kontraktu v Java technologiích. Primárním cílem bylo navrhnout a implementovat nástroj pro extrakci, případně porovnání, konstrukcí DbC ze zdrojových, respektive přeložených, souborů jazyka Java. Zároveň měla být provedena analýza a návrh modelu, který bude schopen takto získaná data reprezentovat. Závěrem bylo za úkol ověřit správnost získaných výsledků a jejich souhrn.\\
	
	V rámci teoretického úvodu bylo třeba čtenáře uvést do problematiky zajištění kvality software, informovat jej o dělení a použití kontraktů a také poskytnout náhled na jazyka Java v oblasti analýzy a dekompilace. Následně byl popsán způsob, jak byl navržen model pro reprezentaci dat a implementace samotného nástroje.\\	
	
	Nástroj byl úspěšně implementován a umožňuje extrakci konstrukcí kontraktů Guava Preconditions a JSR305. Součástí jsou také prostředky pro porovnání těchto konstrukcí a to nejen z hlediska samotných kontraktů, ale také v rámci souborů a celých adresářů. Nástroj lze ovládat pomocí vytvořené uživatelské aplikace, která umožňuje použití příkazů pro rychlé zpracování většího množství souborů či využití grafického rozhraní pro přehlednou práci. Funkčnost obou částí nástroje byla řádně otestována a výsledky zhodnoceny.\\
	
	Všechny stanovené cíle byly úspěšně splněny. Práci je tedy možné použít pro analýzu konstrukcí DbC a dále jí snadno rozšiřovat dle potřeb výzkumu.