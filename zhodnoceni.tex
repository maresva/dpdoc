\chapter{Zhodnocení výsledků}
	Hlavní cíle této práce byl úspěšně zapracovány. Ze vstupních souborů je možné extrahovat konstrukce DbC, které je možné dále exportovat do vnější reprezentace či porovnávat. Funkcionalita tohoto nástroje je pak zakomponována do uživatelské aplikace. Řešení má samozřejmě své silné i slabé stránky a zde bude proveden jejich rozbor.	
	
	
	\section{Úspěšnost detekce kontraktů}
		Na základě provedených testů a analýzy se mi podařilo úspěšně detekovat všechny kontrakty typu JSR305 či Guava Preconditions. Tyto kontrakty byly úspěšně uloženy do navržené datové struktury připraveny pro další zpracování. Kontrakty je možné extrahovat ze zdrojových i přeložených souborů jazyka Java. Kód daného souboru však musí být validní, jinak není možné jej zpracovat analyzátorem.
	
	\section{Úspěšnost porovnání kontraktů}
		Dle provedených testů a zkoumání těchto aktivit si myslím, že jsem při porovnávání dosáhl dobrých výsledků. Je možné porovnávat dvě složky, kde jsou spárovány jednotlivé soubory, třídy, metody a kontrakty. Pro párování kontraktů byla použita heuristika, která s určitými limitacemi umožňuje spojení i kontraktů, které se změnily, či bylo změněno jejich pořadí. Výsledkem porovnávání je seznam změn v rámci kontraktů ale i API obou složek, resp. souborů. Tyto změny je možné dále analyzovat a využít je při zkoumání problematiky použití Design by Contract.\\
		
		Protože oblast párování, nejen kontraktů ale i metod, tříd a souborů, je rozsáhlá a stejně tak to platí pro samotné porovnávání kontraktů, je tyto činnosti možné různě vylepšit. Díky tomu by bylo možné získat více užitečných a přesnějších informací (viz níže).
	 	
 	\section{Prostor pro zlepšení}

		\subsection{Rozpoznání dalších kontraktů}
			Nástroj momentálně umožňuje rozpoznání dvou typů kontraktů (JSR305, Guava Preconditions). To pro použití v praxi samozřejmě nemusí být dostačující a dalším logickým krokem ve vývoji by mělo být přidání rozpoznání dalších specifikací. Nástroj je na to připraven a z hlediska integrace by se tedy nemělo jednat o problém. Obtížnost samotného rozpoznání je pak závislá na povaze dané reprezentace. To samé platí pro párování metod a tříd, kdy změna v signatuře vede k tomu, že tyto objekty nejsou spárovány. Pro párování kontraktů byla vytvořena heuristika, která umožňuje do jisté míry spárovat i kontrakty, které se změnily, či se změnilo jejich pořadí. I přesto, že tato heuristika dosahuje dobrých výsledků, jistě je zde také prostor pro zlepšení.\\
			
			Samotné rozhodnutí které další reprezentace přidat nemusí být triviální. Některé reprezentace může být velmi složité extrahovat a je tak třeba zvážit, zda jsou natolik používané, aby se tato práce vyplatila. K tomuto rozhodování může pomoci článek Contracts in Wild \cite{contractsInWild} i tento text, avšak jistě bude vhodné tyto poznatky rozšířit o další analýzu.
			
		\subsection{Porovnávání}
			Pro porovnávání dvou složek se soubory obsahujícími kontrakty je možné udělat pokročilejší párování souborů. Pokud je soubor pouze přejmenován či přesunut, je to kvalifikováno, jako že byl soubor odstraněn, respektive přidán, díky čemuž dané dva soubory nejsou porovnány. Pro částečné odstranění tohoto problému by bylo možné vytvořit heuristiku, která by porovnávala obsahy souborů, případně by bylo možné dát uživateli volbu v případě neprůkaznosti.\\
			
			Prostorem pro zlepšení je také jednoznačně samotné porovnávání výrazů kontraktů. Momentálně se pouze zkoumá, zda jsou výrazy kontraktů shodné či nikoliv. Toto porovnání však zřejmě může být více komplexní. V případě, že se jedná o logickou podmínku, je možné zkoumat, zda je daný logický výraz shodný i přesto, že je zaměněno pořadí (např. výrazy \texttt{x > 0} a \texttt{0 < x} jsou z logického hlediska shodné, avšak nástroj je vyhodnotí jako rozdílné). Je také možné zaznamenat, zda je některý výraz zpřesňuje či naopak zobecňuje ten druhý. Tyto informace by pak mohly být užitečné při porovnávání různých verzí téhož projektu a určovaní, zda se nezpřísnily kontrakty pro dané API.\\
			
			Pro realizaci tohoto rozšířeného porovnávání by bylo třeba nejprve zjistit, zda daný výraz, obsahuje logický výraz či jiný útvar (může se např. jednat o definici parametru v případě JSR305). Pokud by se jednalo o logický výraz, bylo by jej třeba převést do formy, kde by bylo možné provést další analýzu (např. do stromové struktury). Následně tuto formu porovnat s výrazem svého protějška a vyvést z toho důsledky.
  		
  		\subsection{Efektivita}
	  		Aplikace poskytuje prostor pro zlepšení v rámci efektivity algoritmů. Extrakce porovnání kontraktů se v současné době provádí sekvenčně jak v rámci jednotlivých souborů, tak v rámci typů kontraktů. To má za důsledek delší dobu běhu algoritmu. Tuto činnost by mělo být možné zpracovávat paralelně, což by mohlo celý proces značně urychlit. Samozřejmě by však bylo nutné také přidat ošetření z hlediska konkurence procesů. Alternativně by bylo možné spojit dohromady extrakce jednotlivých typů kontraktů a snížit tak cyklomatičnost celého procesu. Toto zlepšení by však bylo na úkor přehlednosti, jelikož by se museli jednotlivé algoritmy prolínat. Větší efektivita, ale také větší ztráta přehlednosti by byla v případě většího množství rozpoznávaných reprezentací.\\
	  		
	  		Stejně tak to je i v případě porovnávání kontraktů. Zde by však bylo nutné dobře vyřešit párování jednotlivých objektů, jelikož by zde mohlo docházet ke kolizi procesů.
  		
  		\subsection{Uživatelská aplikace}
  			Vzhledem k tomu, že bylo cílem vytvořit pouze jednoduchou uživatelskou aplikaci, která bude primárně sloužit pro další výzkum této problematiky, je v této oblasti mnoho prostoru pro zlepšení. V současné podobně je aplikace vhodná pro zpracování menšího či středního množství souborů. Pro rozšířené použití, by bylo nasnadě zlepšit zobrazení souborů, které by bylo možné např. členit dle projektů, ukládat či jinak zachovávat současné projekty a obecně přidání funkcionalit tohoto typu.\\ 
  			
  			S těmito rozšířeními pro zpracování souborů by také souviselo perzistence zpracovaných dat. Momentálně jsou zpracované soubory uchovávány v paměti, což při malém množství souborů zrychluje práci, nicméně při větším množství souborů se jedná o paměťově náročnou činnost. Zpracované soubory by se tak mohli ukládat např. na pevný disk či do databáze. Díky tomu by také bylo možné se vyhnout opětovné extrakci kontraktů při dalším spuštění aplikace, což by mohlo zrychlit práci v případě, že by se stejnými soubory pracovalo více sezení.\\
  			
  			Pro zlepšení celkové uživatelské zkušenosti by také bylo vhodné doplnit řadu dodatečných akcí, které jsou typické pro programy na správu dat, jak je např. řazení souborů, další možnosti filtrace, ale např. také umožnit použití klávesových zkratek.\\
  			
  			V případě potřeby by také bylo možné zdokonalit konzolovou část aplikace, která by mohla poskytovat dodatečnou funkcionalitu či nastavení.\\
  			
  			Myslím, že díky své silně technické a akademické povaze, není potřebné žádné vylepší vzhledu, které by mohlo být prospěšné např. u komerční aplikace.