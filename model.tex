\chapter{Datový model}
	\section{Rozbor vybraných DbC konstrukcí}
		Po analýze dostupných materiálů jsem se rozhodl zvolit pro implementaci konstrukce Guava Preconditions a JSR305. Důvodem byla především jejich rozdílná reprezentace, kdy Guava Preconditions je realizováno pomocí volání metod uvnitř těl metod a umožňuje vytvářet pre-conditions. Na druhé straně JSR305 je tvořené anotacemi v záhlaví tříd, metod a také jako součást parametrů metod. Umožňuje tvoření všech tří typů kontraktů (pre-conditions, post-conditions, class invariants). Kromě této diverzity se také jedná o jedny z četných konstrukcí používaných v projektech \cite{contractsInWild}.
			
		\subsection{Guava Preconditions}
			Jak již bylo zmíněno, Guava umožňuje tvorbu kontraktů pomocí pre-conditions voláním metod. 
		
		\subsection{JSR305}
	
	\section{Společné znaky}		
		Při rozboru jednotlivých nástrojů pro reprezentaci design by contract snadno zjistíme, že sdílejí mnoho podobných aspektů, které jsou klíčové pro vytvoření obecného modelu, který je schopen zachytit libovolnou konstrukci tohoto kontraktu. 
	
	\section{Vytvořený model}
		\subsection{Popis}
			Výsledný model jsem vytvořil na základě analýzy konstrukcí kontraktů s ohledem na následný export do dat, které bude možné dále zpracovávat. Aby byl zachován kontext kontraktů, usoudil jsem, že bude třeba zachovávat také informace o třídách a metodách v daném souboru. Rozhodl jsem se tedy vytvořit strukturu podobnou stromu, jejíž kořenem je samotný zdrojový soubor. Tento soubor obsahuje různé podrobnosti o tomto souboru jako je jeho jméno a cesta, typ a statistiky o jeho obsahu. Také obsahuje seznam všech tříd obsažených v tomto souboru.
			Každá jednotlivá třída pak obsahuje jméno, svou hlavičku, seznam metod a také seznam všech kontraktů týkajících se této třídy, tedy class invariants. Metoda pak nese informaci o své signatuře a také seznam všech kontraktů této metody.
			
			Samotný kontrakt se pak skládá z těchto informací:
			
			\paragraph{\texttt{ContractType contractType}} Jedná se o výčtový typ, který určuje o jaký druh kontraktu se jedná. Při současném stavu knihovny to mohou být hodnoty Guava či JSR305.
			
			\paragraph{\texttt{ConditionType conditionType}} Opět výčtový typ který určuje typ kontraktu dle jeho podmínky. Rozlišují se tři druhy: pre-condtion, post-condition a class invariant.
			
			\paragraph{\texttt{String completeExpression}} Reprezentuje kompletní výraz celého kontraktu. I přesto, že celý výraz je možné vytvořit z jeho dílčích částí, je zde uveden pro rychlý přehled. Může také posloužit jako kontrola parsování či pro rychlé porovnání.
			
			\paragraph{\texttt{String function}} Tento řetězec určuje funkce o jaký se jedná. V případě Guava se jedná o název metody, v případě JSR305 o název anotace. Obecně se jedná o hlavní označení určující daný typ kontraktu.
			
			\paragraph{\texttt{String expression}} Obsahuje první parametr dané funkce. Důvodem, proč oddělit první parametr od ostatních, bylo, že kontrakty mají často pouze jeden parametr a pokud jich mají více, ostatní často nejsou tolik relevantní. Pro zvýšení přehlednosti byl tedy tento parametr uveden samostatně.
			
			\paragraph{\texttt{List<String> arguments}} Seznam ostatních argumentů daného kontraktu. Ostatní atributy až na výjimky slouží pouze k uvedení chybové zprávy, která se má zobrazit při porušení kontraktu. Mimo zprávy zde také bývají proměnné použité ve zprávě. 
			
			\paragraph{\texttt{String file, className, methodName}} Kontrakt také obsahuje jméno rodičovského souboru, jméno třídy a metody. I přesto, že je tyto atributy možné získat od rodičovských objektů, jsou zde z důvodů přehlednosti exportu a zejména pak kvůli porovnávání, kde urychlují celý proces.
		
		\subsection{Diagram}
			Obrázek
			
		\subsection{Reprezentace modelu}
			Po dohodě s vedoucím práce jsem se rozhodl pro externí reprezentaci modelu použít formát JSON. Vzhledem k tomu, že JSON je široce používaný zápis dat a umožňuje relativně snadné ukládaní objektů typu Java a také umožňuje další zpracování. JSON je tak vhodný pro zpracovaní strojem, ale je dobře čitelný i pro lidské oko (v případě, že byl zformátován).