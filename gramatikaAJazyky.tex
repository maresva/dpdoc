\chapter{Analýza kódu jazyka Java}

	\section{O jazyce Java}
		Java \cite{javaEnviroment} je objektově orientovaný programovací jazyk, který byl vytvořen více než před 20 lety. Java byla inspirována řadou jazyků, jako je např. Eiffel, SmallTalk a Objective C. Snahou bylo vytvořit objektově orientovaný jazyk, který bude jednoduchý na pochopení, aniž by bylo třeba dlouhého tréninku. Jedním ze značných usnadnění, např. oproti jazyku C/C++, je Garbage Collector. Ten umožňuje automatické uvolňování paměti, což předchází častým chybám spojených s jejím manuálním uvolňováním.	Cílem také bylo, aby byla Java robustní a zabezpečená a jednalo se tak o spolehlivý jazyk. Java byla vytvářena za účelem architektonické a platformní nezávislosti a bylo ji tak možné použít takřka všude. Této přenositelnosti bylo zajištěno především tím, že se jedná o interpretovaný jazyk (viž níže). Při vývoji bylo také samozřejmě cíleno na zajištění co největší výkonnosti s čímž souvisí i umožnění tvorby vícevláknových aplikací.\\
		
		Java má řadu výhod, ale také nevýhod a stejně jako u každé jiné technologie je třeba zvážit, zda se jedná o vhodnou volbu pro náš projekt. Jazyk Java je dlouhodobě jedním z nejpoužívanějších programovacích jazyků a stále se vyvíjí \cite{javaTrend}.\\
		
		Tato kapitola bude čtenáře informovat o základních vlastnostech jazyka Java, především pak o jeho překladu do bajtkódu. Jejím hlavním cílem je představení gramatiky tohoto jazyka a úvod do problematiky tokenizace, která v důsledku umožní extrakci konstrukcí DbC. Dalším tématem budou také možnosti dekompilace bajtkódu a rozbor toho, jak se výsledek liší oproti zdrojovému kódu.
		
		
		\subsection{Kompilace jazyka Java}	
			Jazyk Java je tzv. interpretovaný jazyk, což znamená, že programovací kód není překládán přímo do strojového kódu daného zařízení, ale je přeložen do bajtkódu. Bajtkód je speciální vysoce-úrovňový kód, který je platformově nezávislý. V rámci kompilace Java to znamená, že zdrojové soubory \texttt{.java} jsou překompilovány do souborů \texttt{.class}. Výsledný Java program je pak distribuován ve formátu \texttt{.jar} nebo \texttt{.war}, což jsou prakticky archivy obsahující přeložené soubory. Na cílovém zařízení je pak program spouštěn pomocí \emph{Java Virtual Machine} (Virtuální stroj jazyka Java, dále JVM).
			
			\subsubsection{Java Virtual Machine (JVM)}			
				JVM je softwarová abstrakce pro obecnou hardwarovou platformu. Slouží k tomu, aby bylo možné spustit program napsán v Java na různých zařízeních. Program je pak virtuálně spuštěn na JVM místo přímo na daném zařízení. Vzhledem k tomu, že se jedná o virtualizaci, je logické, že z hlediska výkonu a nároků na paměť není možné dosáhnout srovnatelných výsledků s programem, který je na danou platformu plně zoptimalizován. V tomto případě se jedná o daň za multiplatformnost jazyka Java.\\ 
			
			Když spustíme Java aplikace, nejprve se spustí JVM na daném zařízení. Ten načte hlavní třídu s metodou \texttt{main} spolu s jinými klíčovými částmi jako je např. \texttt{java.lang.Object}. K načtení těchto tříd se využívá tzv. \emph{Class Loader}. Poté, co jsou načteny klíčové části již JVM načítá třídy pouze na vyžádání, což se děje ve chvíli, kdy je daná třídy v programu potřeba. Tímto způsobem obsahuje JVM jen ten přeložený kód, který je v danou chvíli potřeba, což snižuje nároky na paměť, ale snižuje rychlost programu. Tato technika se nazývá \emph{Just-in-time kompilace} \cite{jit}\cite{classLoad}.
			
			\textbf{\textcolor{pblue}{TODO: JIT je něco jiného}}\\
				
	
	\section{Gramatika}
		\textbf{\textcolor{pblue}{TODO: reprezentace některých konstrukcí, vč. anotací - do jakých základních konstrukcí jsou převáděny, jak se poznají v bytecode; retention policy u anotací.	}}\\
	
	\section{Rozbor kódu}
		Abychom mohli zpracovávat zdrojový kód jazyka Java, je vhodné jej nejprve převést na snadněji zpracovatelnou formu než surový text. K tomu nám může pomoci lexikální a syntaktická analýza, pomocí nichž můžeme ve výsledku získat reprezentaci ve stromové podobě \cite{compilerTutorial}.
	
		\subsection{Lexikální analýza}
		
			\textbf{\textcolor{pblue}{TODO: toto je hodně volný a nepřesný popis lex a syn analýzy - nutno zpřesnit, viz aspoň FJP!}}\\					
		
			Aby bylo možné provést rozbor kódu, který je prezentován v textové podobě, je třeba nejprve provést lexikální analýzu. Ta spočívá v tom, že je proud znaků zpracováván a rozdělován dle bílých znaků\footnote{Bílými znaky jsou myšleny netisknutelné znaky jako jsou například mezery, tabulátory, nové řádky atd.} nebo operačních, respektive speciálních, symbolů (\texttt{=}, \texttt{:}, \texttt{\&}, ...) na tzv. lexémy, které reprezentují \textbf{\textcolor{pblue}{dílčí části kódu}}. Na základě sekvence znaků analyzátor pomocí regulárních výrazů určí, zda se jedná o identifikátor, číslo, klíčové slovo, řetězec atd. Pokud analyzátor narazí na nesrovnalost, např. v Java by začal lexém číslem a pokračoval písmeny, což by představovalo ilegální pojmenování identifikátoru, analýza vyhodí chybu. Tímto vznikne seznam lexémů, který je předán dál syntaktické analýze.
			
		\subsection{Syntaktická analýza}
			Ze vstupního proudu lexému vytvořeného pomocí lexikální analýzy můžeme nyní za pomocí syntaktické analýzy vytvořit derivační strom. Díky této reprezentaci \textbf{\textcolor{pblue}{můžeme již dané lexémy rozdělit do různých konstrukcí jazyka}}, jako jsou třídy, metody, podmínky atd. Po tomto kroku jsme již tedy schopni zpracovávat zdrojový kód na takové úrovni, která nám umožní procházet anotace jednotlivých konstrukcí, či příkazy těla metod. Na základě toho jsme tedy schopni provádět analýzu, která umožní nalezení kontraktů ve zdrojovém kódu. 			
			
		\subsection{Nástroje}
			I přesto, že by bylo možné provádět zmíněné analýzy pomocí vlastního kódu, je také možné použít některý z dostupných nástrojů. Tyto nástroje umožňují parsování zdrojového kódu za použití samostatné aplikace nebo se může jednat o knihovnu, jejíž API můžeme použít přímo v našem kódu.
			
			\paragraph{JFLex}
				JFLex \cite{jflex} je lexikální analyzátor pro jazyk Java. Protože lexikální analýza sama o sobě nestačí k analýze kódu vhodné pro detekci kontraktů, je třeba tento nástroj použít s jiným, který umožní syntaktickou analýzu. Může se jednat např. o CUP, BYacc/J či ANTLR (viz níže). 						
			
			\paragraph{CUP}
				TODO
			
			\paragraph{BYacc/J}
				TODO
			
			\paragraph{ANTLR}			
				ANTLR \cite{antlr} je široce používaný nástroj, který umožňuje na základě zadané gramatiky vytvořit a procházet strom daného zdrojového kódu. ANTLR je primárně určen ke zpracování vlastní gramatiky např. při tvorbě vlastního překladače, nicméně je možné také použít gramatiku jazyka Java pro rozbor jejího zdrojového kódu.
			
			\paragraph{JavaParser}
				JavaParser \cite{javaparser} je jedním z dostupných nástrojů, který umožňuje analýzu, parsování ale i konstruování kódu jazyka Java. Funguje jako knihovna, kterou můžeme použít v našem kódu. Jedná se o vyspělou technologii, která je použita v řadě projektů. 
				
			\paragraph{Roaster}
				Roaster \cite{roaster} je nástroj podobný JavaParser. Také umožňuje nejen parsování, ale i editaci zdrojových kódů jazyka Java. Mimo dostupného API umožňuje také práci s konzolí  pro jednoduché použití.
	
	
	\section{Rozbor přeložených souborů}
		Při analýze kontraktů v jiných projektech není vždy možné získat zdrojové soubory. Z toho důvodu je užitečné umět analyzovat nejen soubory \texttt{.java}, ale také přeložené soubory \texttt{.class}.
	
		\subsection{Dekompilace}
			Aby bylo možné soubory analyzovat je nutné je nejprve dekompilovat. Dekompilace je vlastně opačný proces kompilace a je třeba nahradit prvky, které byly přidány do bajtkódu těmi, které zde byly původně. \textbf{\textcolor{pblue}{TODO: Zlepšit formulaci, více podrobností}}\\
			
			\subsubsection{Nástroje}
				Nástrojů pro dekompilaci souborů v jazyce Java je celá řada. Některé je nutné pouštět samostatně, ty pak mohou být vytvořeny i v jiných programovacích jazycích. Vývojová prostředí jako např. Eclipse či IDEA IntelliJ umožňují instalaci balíčků, které podporují dekompilaci souborů. Některé z nástrojů je také možné použít přímo v kódu formou knihovny. Následuje seznam některých dostupných nástrojů, které jsou stále aktualizovány a umožňují dekompilaci moderních prvků jazyka Java \cite{top8decompilers}\cite{decompilersOnline}\cite{quickDecompilers}.
				
				\textbf{\textcolor{pblue}{TODO:
				chybí podrobnější informace, do jaké reprezentace je bytecode daným nástrojem převeden a jestli/jak se s ní dá programově pracovat (tj. doménové a funkční API);, vč. příkladů; dále jaká jsou omezení či jiné pozoru-hodné vlastnosti nástrojů	 }}\\			
				
				\paragraph{JD Project}
					JD Project (Java Decompiler) \cite{jd} je jedním z nejčastěji referovaných nástrojů v rámci Java dekompilace. JD disponuje samostatnou aplikací s grafickým uživatelským rozhraním, pomocí kterého je možné soubory dekompilovat a ihned vidět výsledek. JD také poskytuje doplněk do Eclipse a IDEA IntelliJ. Nejedná se o open-source nástroj, ale je dostupný zdroj pro GUI aplikaci.
					
				\paragraph{Procyon}
					Procyon \cite{procyon} je open-source nástroj, který také umožňuje dekompilaci. Nemá samostatné GUI jako JD, nicméně je možné reprezentaci zobrazit použitím externích nástrojů. Disponuje také API, které je možné použít přímo v kódu.
					
				\paragraph{CFR}
					CFR \cite{cfr} je dalším z nástrojů, který umožňuje dekompilaci i konstrukcí Java 1.9. Nástroj je možné použít pouze v rámci příkazové řádky.\\
					
				Nástrojů pro dekompilaci jazyka Java je velké množství a jejich výběr není snadný zejména poroto, že se neustále vyvíjejí na základě nových verzí Java. Při výběru je samozřejmě také důležité, jakým způsobem je daný nástroj možné používat, jaké konstrukce dokáže dekompilovat a roli může hrát také rychlost.
				
			\subsubsection{Rozdíly oproti původním souborům}
				Jak již bylo zmíněno, velice závisí na použitém nástroji pro dekompilaci. Jestliže daný nástroj nepodporuje určité konstrukce, pak je logické že jejich dekompilace nebude možná a není tak možné získat původní zdrojový soubor. Pokud nástroj umožňuje dekompilaci všech použitých konstrukcí, výsledný kód stále nebude zcela shodný. Přeložený kód obsahuje plné cesty k objektům, doplňuje neuvedená přetypování atd. Tyto informace často ve zdrojových kódech nebývají uvedené, protože jsou redundantní, ale při rekonstrukci přeloženého souboru není možné určit, kde tyto informace byly a kde ne. Přirozeně také není možné získat data, která se do bajtkódu nezanášejí, jako jsou např. komentáře.