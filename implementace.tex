\chapter{Implementace nástroje pro analýzu kontraktů}

	shrnout zadání (kontext, proč se ta aplikace dělá, pak návrh řešení) 

%%%%%%%%%%%%%%%%%%%%%%%%%%%%%%%%%%%%%%%%%%%%%%%%%%%%%%%%%%%%%%%%%%%%%%%%%%%%%%%%%%%%%%%%%%%%%%%%%%%%%%%%%%%%%%%%%%%%%%%%%%%%%%%%%%%%%%%%%%%% 
	\section{Implementace knihovny}
	    \subsection{Použité technologie}
			Knihovna byla implementována v jazyce Java verze 1.8 ve vývojovém prostředí IDEA IntelliJ Ultimate 2017.3.3. Pro zajištění snadného získání závislostí a následného zjednodušení použití knihovny byla pro vytvoření projektu použita technologie Apache Maven.
			
		\subsubsection{Knihovny třetích stran}
			Při vývoji knihovny pro analýzu kontraktů byly použity následující knihovny třetích stran:
			
			\paragraph{Apache Log4j} 
				Tato knihovna umožňuje pokročilé možnosti pro logování. Byla použita zejména pro zaznamenávání chyb a různých informačních záznamů \cite{log4j}.
				
			\paragraph{Procyon}
				Knihovna Procyon byla použita pro dekompilaci přeložených Java souborů \texttt{*.class} \cite{procyon}.
				
			\paragraph{JavaParser}
				JavaParser byl použit tokenizaci zdrojových souborů \cite{javaparser}. 
				
			\paragraph{Google Gson}
				Tato knihovna poskytuje prostředky pro uložení objektů jazyka Java do reprezentace pomocí formátu JSON \cite{gson}. 
				
			\paragraph{jUnit 5}
				Poskytuje možnosti testování pomocí jednotkových testů \cite{junit}.    	 
		
		
		\subsection{Dekompilace Bytcode}
			Pro dekompilaci Java \texttt{*.class} souborů byla použita knihovna Procyon. Ta umožňuje použití metody \texttt{decompile()}, která přečte vstupní soubor s přeloženým kódem a do jiného souboru uloží jeho dekompilovanou verzi. Tento dočasný soubor s dekompilovaným kódem je poté předán pro zpracování třídě \texttt{JavaFileParser}, která jej zpracuje stejně jako běžný zdrojový soubor. V knihovně dekompilaci obstarává metoda \texttt{decompileClassFile()}, která se nachází v třídě \texttt{io.IOServices}.
					
		    	 
	    \subsection{Parsování Java souborů}	    
			Pro zpracování zdrojových souborů jazyku Java byla použita knihovna JavaParser. Ta poskytuje metodu \texttt{parse()}, která vytvoří komplexní strukturu daného zdrojového souboru. V prvním kroku se tato struktura projde a vyhledá všechny třídy (\emph{class}) a také rozhraní \emph{interface} a výčtové typy \emph{enum}. Pro účely modelu jsou si tyto tři prvky rovny. Každý nalezený prvek je následně uložen do modelu. V případě třídy a rozhraní je se struktura prochází dále a do modelu jsou uloženy všechny konstruktory, které se z hlediska modelu považují za metody (viz níže). Následně jsou uloženy všechny anotace dané \uv{třídy}.
			
			Po této přípravě je využita třída \texttt{MethodVisitor}, která dědí od třídy \texttt{VoidVisitorAdapter} a umožňuje procházet všechny metody v daném souboru. V metodě pak máme k dispozici objekt typu \texttt{MethodDeclaration}, který obsahuje všechny potřebné údaje a také rodičovský \texttt{ExtendedJavaFile}. Pro každou metodu je nalezena její rodičovská třída. Hledá se nejvyšší rodič a tudíž vnořené metody nemají jako rodiče vyšší metodu ale nejvyšší dostupnou třídu. Pro danou metodu jsou následně uloženy všechny anotace a i její parametry. Následně je uloženo celé tělo metody jako seznam objektů typu \texttt{Node}, které umožňují další zpracování. Z těchto získaných dat je vytvořena instance objektu \texttt{ExtendedJavaMethod}, která je následně uložena do 
		
		\subsection{Extrakce kontraktů}
			
			\subsubsection{Obecně}
			\subsubsection{Guava Precondtions}
			\subsubsection{JSR305}
		
	    \subsection{Porovnávání kontraktů}
	    
	    \subsection{Popis API}
	    
	    \subsection{Přidání parseru pro nový typ kontraktu}
    
%%%%%%%%%%%%%%%%%%%%%%%%%%%%%%%%%%%%%%%%%%%%%%%%%%%%%%%%%%%%%%%%%%%%%%%%%%%%%%%%%%%%%%%%%%%%%%%%%%%%%%%%%%%%%%%%%%%%%%%%%%%%%%%%%%%%%%%%%%%%
	\section{Implementace uživatelské aplikace}
	   \subsection{Použité technologie}
	    	 Java, JavaFX, Maven, knihovny, ...	
	    	   
	   \subsection{Rozdělení aplikace}
	   		\subsubsection{Grafická část}
	   		\subsubsection{Konzolová část}
	   
	   \subsection{Možnosti a limitace aplikace}